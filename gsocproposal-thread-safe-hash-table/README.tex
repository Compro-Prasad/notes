% Created 2017-03-26 Sun 03:06
% Intended LaTeX compiler: pdflatex
\documentclass[11pt]{article}
\usepackage[utf8]{inputenc}
\usepackage[T1]{fontenc}
\usepackage{graphicx}
\usepackage{grffile}
\usepackage{longtable}
\usepackage{wrapfig}
\usepackage{rotating}
\usepackage[normalem]{ulem}
\usepackage{amsmath}
\usepackage{textcomp}
\usepackage{amssymb}
\usepackage{capt-of}
\usepackage{hyperref}
\author{Compro Prasad}
\date{\today}
\title{}
\hypersetup{
 pdfauthor={Compro Prasad},
 pdftitle={},
 pdfkeywords={},
 pdfsubject={},
 pdfcreator={Emacs 25.1.1 (Org mode 9.0.5)},
 pdflang={English}}
\begin{document}

\section{Name}
\label{sec:org7b82c92}
Abhishek Prasad
\section{Your email address}
\label{sec:orgf6de055}
comproprasad@gmail.com
\section{The name of the project}
\label{sec:orge7eba49}
GNU Clisp
\section{Summary}
\label{sec:org9b5ca96}
In simple terms I will be implementing thread safe hash tables in the core of
GNU Clisp.
\section{Benefits}
\label{sec:org52e1722}
Having threading support is necessary everywhere from normal computing tasks to
larger applications running on production environments like servers. The problem
arises when the data being used in threads are not atomic in nature. Most of
data structures are thread safe in nature except the hash tables which are also
extensively used in \href{https://sourceforge.net/p/clocc/hg/ci/default/tree/}{CLOCC}.

Since I am doing this project I will get to know how threads, mutexes and hash
tables are used in writing a program like GNU Clisp. And overall I would be
communicating with great people in the GNU community I would learn other side
stuffs too which will be an additional gift for me.
\section{Things to be delivered}
\label{sec:org73d8522}
Since, my work is consists of threading and hash table so the files to be dealt
with are \texttt{src/zthread.d}, \texttt{src/xthread.d} and \texttt{src/hashtabl.d} and may be some
of the related lisp files.

Leapfrog hash tables are faster to probe so they will probably be used
\section{Plan}
\label{sec:org5c908e1}
\begin{center}
\begin{tabular}{ll}
Date & Work\\
\hline
04-May to 30-May & Talking to the developer team, fixing small bugs and\\
 & get acquainted with the project\\
 & \\
31-May to 11-June & Writing the lisp code to make simple threads and find\\
 & out where the hash table hits a corner case. Get ready\\
 & with enough tests.\\
 & \\
09-June to 30-June & Write functions to handle the thread safety in the\\
 & hash tables.\\
 & \\
25-June to 15-July & Have the functions run against the test cases\\
 & \\
10-July to 20-July & Write the tests that were missed beforehand\\
 & \\
18-July to 25-July & Make the code more portable to run on different\\
 & platforms\\
 & \\
23-July to 30-July & Polishing the code to be more readable for future\\
 & development\\
\end{tabular}
\end{center}
\section{Communication}
\label{sec:org4ae0a08}
\href{https://sourceforge.net/p/clisp/mailman/clisp-devel/}{clisp-devel} mailing list archives on source forge is subscribed both via email
and RSS feeds. \href{https://sourceforge.net/p/clisp/mailman/clisp-announce/}{clisp-announce} is the place where announcements are made.
User bugs are available on \href{https://sourceforge.net/p/clisp/mailman/clisp-list/}{clisp-list}.

I have also asked some questions to the GNU Clisp developers who have helped me
most of the time but I am loving the new things that I learn when I communicate.
\section{Qualification}
\label{sec:orgfd02e25}
I have written a mathematical expression calculator and have optimized as much
as possible. So, I am well acquainted with programming in C/C++ using Makefiles.
I also like designing/programming core stuff like a compiler or a kernel or a
DBMS.

Since, I am an Emacs user for around a year so I have started to like ELisp and
gradually lisp too.

\begin{itemize}
\item National Institute of Technology, Durgapur
\item Bachelor's course in Information Technology
\item Second(2nd year or 4th semester) year
\item Multi threading concept is in 4th semester
\item Hash Tables were taught in the 3rd semester
\item Basics of \texttt{pthread.h}
\item Written a mathematical expression calculator
\item Makefiles
\item Bash
\end{itemize}
\end{document}
