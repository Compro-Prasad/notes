% Created 2016-10-27 Thu 12:15
\documentclass[11pt]{article}
\usepackage[utf8]{inputenc}
\usepackage[T1]{fontenc}
\usepackage{fixltx2e}
\usepackage{graphicx}
\usepackage{grffile}
\usepackage{longtable}
\usepackage{wrapfig}
\usepackage{rotating}
\usepackage[normalem]{ulem}
\usepackage{amsmath}
\usepackage{textcomp}
\usepackage{amssymb}
\usepackage{capt-of}
\usepackage{hyperref}
\author{Compro Prasad}
\date{\today}
\title{Socket Programming}
\hypersetup{
 pdfauthor={Compro Prasad},
 pdftitle={Socket Programming},
 pdfkeywords={},
 pdfsubject={},
 pdfcreator={Emacs 24.5.1 (Org mode 8.3.6)}, 
 pdflang={English}}
\begin{document}

\maketitle
\tableofcontents

\section{Create a socket}
\label{sec:orgheadline1}
\begin{verbatim}
int socket_desc = socket(AF_INET, SOCK_STREAM, 0);

if (socket_desc == -1) {
	printf("Could not create socket");
}
\end{verbatim}
The above code will create a socket with following properties:
\begin{verbatim}
Address Family - AF_INET (this is IP version 4)
Type - SOCK_STREAM (this means connection oriented TCP protocol)
Protocol - 0 [ or IPPROTO_IP This is IP protocol]
\end{verbatim}
\section{Connect socket to a server}
\label{sec:orgheadline6}
We connect to a remote server on a certain port number. So we need
2 things, \textbf{ip address} and \textbf{port number}.
\subsection{\texttt{sockaddr\_in} structure}
\label{sec:orgheadline2}
To connect to a remote server firstly we create \texttt{sockaddr\_in}
structure with proper values.
\begin{verbatim}
struct sockaddr_in server;
\end{verbatim}
The structure definations are as follows:
\begin{verbatim}
// IPv4 AF_INET sockets:
struct sockaddr_in {
	short          sin_family;    // e.g. AF_INET, AF_INET6
	unsigned short sin_port;      // e.g. htons(3490)
	struct in_addr sin_addr;      // see struct in_addr, below
	char           sin_zero[8];   // zero this if needed
};

struct in_addr {
	unsigned long s_addr;         // load with inet_pton()
};

struct sockaddr {
	unsigned short sa_faminly;    // address family, AF_xxx
	char           sa_data;       // 14 bytes of protocol address
}
\end{verbatim}
\texttt{s\_addr} of \texttt{in\_addr} structure will contain the \textbf{IP address} in \uline{long format}.
\subsection{\texttt{inet\_addr} function}
\label{sec:orgheadline3}
To convert an \textbf{IP address} to a \uline{long format} \texttt{inet\_addr(const char *)}
function is used.

For example,
\begin{verbatim}
server.sin_addr.s_addr = inet_addr("176.34.135.167");
\end{verbatim}
can be used for connecting to \href{http://176.34.135.167}{DuckDuckGo} search engine where \texttt{176.34.135.167} is
the \textbf{IP address} passed as a \texttt{string} parameter.
\subsection{\texttt{connect} function}
\label{sec:orgheadline5}
\texttt{connect} is a function for connecting to a remote server. A sample code is
given below:
\begin{verbatim}
server.sin_family = AF_INET;
server.sin_port = htons(80);

// Connect to remote server
if (connect(socket_desc, (struct sockaddr *)&server, sizeof(server)) < 0) {
	puts("connect error\n");
	return 1;
}

puts("Connected\n");
\end{verbatim}
So, we have \textbf{created} a socket and \textbf{connected} it to a server. Now we are going
to \textbf{send} / \textbf{transmit} data to the remote server.
\subsubsection{Note: \textbf{Connections are present only in TCP sockets}}
\label{sec:orgheadline4}
Concept of '\textbf{connections}' apply to \texttt{SOCK\_STREAM} / \texttt{TCP} type of sockets. Connection
means a reliable '\textbf{stream}' of data such that there can be multiple such streams
communication of its own. It can be considered a pipe \textbf{not interfered} by other data.

UDP(\href{https://en.wikipedia.org/wiki/User_Datagram_Protocol}{User Datagram Protocol}), ICMP(\href{https://en.wikipedia.org/wiki/Internet_Control_Message_Protocol}{Internet Control Message Protocol}), ARP(\href{https://en.wikipedia.org/wiki/Address_Resolution_Protocol}{Address Resolution Protocol})
are \textbf{non-connection} based communication which means packets can be sent to anybody
and everybody.
\section{Sending data over socket}
\label{sec:orgheadline8}
\subsection{\texttt{send} function}
\label{sec:orgheadline7}
It needs the \textbf{socket descriptor} returned after creating a socket, the \textbf{data to
send} and \textbf{its size}. We have the following code which sends data to \href{http://176.34.135.167}{DuckDuckGo}.
\begin{verbatim}
// Send some data
char *message = "GET /HTTP/1.1\n";
if (send(socket_desc, message, strlen(message), 0) < 0) {
	puts("Send failed\n");
	return 1;
}
puts("Data sent\n");
\end{verbatim}
The \texttt{message} string is actually commanding the server to \textbf{get} the mainpage
of a website.\\
In the next section we try to recieve a reply from the server.
\section{Recieve data on socket}
\label{sec:orgheadline11}
\subsection{\texttt{recv} function}
\label{sec:orgheadline10}
The \texttt{recv} function will try recieving data through socket from a web server.
\begin{verbatim}
// Receive a reply from the server
char server_reply[2000];
if (recv(socket_desc, server_reply, 2000, 0) < 0) {
	puts("recv failed\n");
}
puts("Reply received\n");
puts(server_reply);
\end{verbatim}
\subsubsection{Note:}
\label{sec:orgheadline9}
When receiving data on a socket, we are basically \textbf{reading} it. This is similar
to reading data from a file(remember the Unix philosophy?). So we can use the
\texttt{read} function to read data on a socket. For example:
\begin{verbatim}
read(socket_desc, server)
\end{verbatim}
\section{Closing a socket}
\label{sec:orgheadline12}
Just like files, sockets also need to be closed. We can use the primary \texttt{close}
function which accepts a \textbf{file descriptor} as an argument.
\begin{verbatim}
close(socket_desc);
\end{verbatim}
\end{document}