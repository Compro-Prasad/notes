% Created 2016-10-31 Mon 16:42
\documentclass[11pt]{article}
\usepackage[utf8]{inputenc}
\usepackage[T1]{fontenc}
\usepackage{fixltx2e}
\usepackage{graphicx}
\usepackage{grffile}
\usepackage{longtable}
\usepackage{wrapfig}
\usepackage{rotating}
\usepackage[normalem]{ulem}
\usepackage{amsmath}
\usepackage{textcomp}
\usepackage{amssymb}
\usepackage{capt-of}
\usepackage{hyperref}
\author{Compro Prasad}
\date{\today}
\title{Structured Query Language(SQL)}
\hypersetup{
 pdfauthor={Compro Prasad},
 pdftitle={Structured Query Language(SQL)},
 pdfkeywords={},
 pdfsubject={},
 pdfcreator={Emacs 24.5.1 (Org mode 8.3.6)}, 
 pdflang={English}}
\begin{document}

\maketitle
\tableofcontents

\section{Introduction}
\label{sec:orgheadline1}
\texttt{SQL} is the interface for communicating to a binary file
which is called a \texttt{database} in terms of \texttt{SQL}. So, keep in
mind that if you change the binary file using an editor like
\href{http://hexedit.com/}{hexedit} then you could run into problems. So, don't mess
with such kinds of files on your computer.\\
A \textbf{database} is further divided into \textbf{tables}. Tables are further
divided into two categories: \textbf{columns} and \textbf{rows}.\\
Please make sure a \texttt{SQL} is installed on your machine before
starting this tutorial.
\section{Basics}
\label{sec:orgheadline10}
\subsection{Logging in}
\label{sec:orgheadline2}
It is necessary to login to access the whole/part of your
\texttt{SQL}. Now you would say that there is no use of login. But
what if the frontend developer of \href{https://facebook.com}{Facebook} gets into the
database and steals information of everybody.\\
So, there is a concept of users in \texttt{SQL}. This removes the
necessity of having different things in the same place.
Instead, \texttt{SQL} just puts everything in the same place and
permits the specific user to the specific information.

Here, is a simple command for logging into \texttt{SQL} using
\textbf{tom} as the username:
\begin{verbatim}
mysql -u tom -p
\end{verbatim}
After pressing \texttt{RETURN} mysql would ask for a password.
Enter your password and you are good to go.

Note: During the time of installation the default user is \textbf{root}.
\subsection{What is a query?}
\label{sec:orgheadline3}
Query is a successful statement in \texttt{SQL} that gets a job done
by the end user. It can be creating tables or inserting values
or modifying values etc. The following are some examples of
queries in \texttt{MySQL}:
\begin{verbatim}
create table information(column1 int);
use login_db;
select 1+2 from dual;
\end{verbatim}
\subsection{Accessing Databases/Schemas}
\label{sec:orgheadline4}
Now, since you have logged in you have the control over what
you have access to. You may not have been permitted to access
even a single database. To see the schemas run this query:
\begin{verbatim}
show databases;
\end{verbatim}
\subsection{Accessing Table}
\label{sec:orgheadline6}
Don't you know tables?!
There can be many tables inside a database. Thats enough
introduction to tables in \texttt{SQL}
\begin{verbatim}
show tables from information_schema;
\end{verbatim}
Replace \textbf{information\(_{\text{schema}}\)} with the database of your
choice.\\
You can use:
\begin{verbatim}
show tables;
\end{verbatim}
when you are using a specific database. See \hyperref[sec:orgheadline5]{Using databases}.
\subsection{Accessing columns}
\label{sec:orgheadline8}
\begin{verbatim}
show columns from information_schema.collations;
\end{verbatim}
It is of the form \texttt{<database>.<table>}.\\
Want a shorter form? Use the following approach:
\begin{verbatim}
describe information_schema.collations;
\end{verbatim}
\subsubsection{Most used types of columns:}
\label{sec:orgheadline7}
\begin{center}
\begin{tabular}{ll}
Type & Description\\
\hline
\texttt{char(n)} & A string of characters of length \texttt{n}.\\
 & Same memory consumption for different\\
 & strings.\\
\texttt{varchar(n)} & A string of characters of length \texttt{n}.\\
 & Memory consumption according to length\\
\texttt{int} & An integer field\\
\texttt{bigint} & An integer field with a large range\\
\texttt{date} & A date of the form \texttt{YYYY-MM-DD}\\
\texttt{datetime} & A field addressing a date and time\\
 & value.\\
\end{tabular}
\end{center}
\subsection{Using databases}
\label{sec:orgheadline5}
Databases are less in comparison to the tables in
each database. Therefore, writing \texttt{<database>.<table>}
can be a bit tedious. So, there is a feature in sql
and that is using a database:
\begin{verbatim}
use information_schema;
\end{verbatim}
You can now see columns using:
\begin{verbatim}
describe collations;
\end{verbatim}
Since you are using the database, you cannot operate
on a table thats inside another database using a normal
syntax.
But you can do two things outside the database. 
They are:
\begin{verbatim}
show databases;
use performance_schema;
\end{verbatim}
Here \textbf{performance\(_{\text{schema}}\)} can be another database you
would like to use.
From now on it will be assumed that you are using a
database and then the respective query can be run after
that.
\subsection{Accessing rows}
\label{sec:orgheadline9}
See all rows and columns from a table:
\begin{verbatim}
select * from collations;
\end{verbatim}
Filter specific set of rows from a table:
\begin{verbatim}
select is_default, sortlen from collations;
\end{verbatim}
Filter specific set of rows and columns from a table:
\begin{verbatim}
select is_default, sortlen from collations where id < 50 and is_default='Yes';
\end{verbatim}
\section{Creating}
\label{sec:orgheadline14}
\subsection{Databases}
\label{sec:orgheadline11}
\begin{verbatim}
create database login_db;
\end{verbatim}
\subsection{Tables}
\label{sec:orgheadline13}
\subsubsection{Basic table}
\label{sec:orgheadline12}
\end{document}